\documentclass[11pt]{article}
\newcommand{\itt}[1]{\textit{#1}}
\newcommand{\bo}[1]{\textbf{#1}}
\newcommand{\bi}{\begin{itemize}}
\newcommand{\ii}{\item}
\newcommand{\ei}{\end{itemize}}
\newcommand{\ms}{\medskip}
\usepackage[margin=1in]{geometry}
\usepackage{enumitem}
\usepackage{hyperref}


%\pagestyle{empty}
%\setlength{\oddsidemargin}{1cm}
%\setlength{\evensidemargin}{1cm}
%\address{3855 West 41st Ave.,\ \\ Vancouver BC, \\ V6N 3E9.}
%\signature{Eric Rosen}
\begin{document}
\begin{center}
\LARGE
Curriculum Vitae
\normalsize
\end{center}
\medskip

\noindent Eric Rosen \\
%3855 West 41st Ave., \\
%Vancouver, B.C. \\
%V6N 3E9 \\
\verb+errosen@mail.ubc.ca+ \\
\verb+errosen@shaw.ca+ \\
Personal website: \verb+https://stanleythepug.github.io/+\\
%\verb+erosen27@jh.edu+ \\
%\verb+b-erosen@microsoft.com+ \\
%tel. 1-604-442-7889 \\

\noindent \textbf{Recent research position}
\bi
\ii[] Visiting postdoctoral fellow, Department of Cognitive Science, Johns Hopkins University
\ii[] Duration of contract: February 7th, 2018 - January 31st, 2021
\ii[] Faculty sponsor: Dr.\ Paul Smolensky, Krieger-Eisenhower Professor, with cross appointment at Microsoft Research
\ei

\noindent \textbf{Current research position}
\bi
\ii[] Visiting postdoctoral fellow, Institute of Linguistics, University of Leipzig
\ii[] Duration of contract: November 1st, 2021 - March 31st, 2023.
\ii[] Faculty sponsor: Dr.\ Jochen Trommer, Professor
\ei

\noindent \bo{Research publications since 2016}

\begin{enumerate}[label={[\arabic*]}]
\ii to appear on January 31, 2022.  \textit{Modeling human-like morphological prediction}.  To appear in the Proceedings of the 2022 meeting of the Society for Computation in Linguistics.
\ii to appear (with second author Matthew Goldrick) \textit{Interaction of lexical strata in hybrid compound words through gradient phonotactics}. To appear in the 2022 Supplemental Proceedings of the Annual Meeting on Phonology, Toronto. October 3, 2021.
\ii 2021. \href{https://scholarworks.umass.edu/scil/vol4/iss1/14/}{\itt{Inflectional paradigms as interacting systems}} Proceedings of the 2021 meeting of the Society for Computation in Linguistics.
\ii 2021. \href{https://scholarworks.umass.edu/scil/vol4/iss1/49/}{\itt{Lexical strata and phonotactic perplexity minimization}}. Supplemental proceedings of the Annual Meeting on Phonology, UC Santa Cruz, September 2020. Also presented as a poster and extended abstract at the 2021 meeting of the Society for Computation in Linguistics.
\ii 2021. (Caitlin Smith, Charlie O'Hara Paul Smolensky and Eric Rosen) \\\href{https://caitlinsmith14.github.io/pdf/smithetal_scil2021_paper.pdf}{\itt{Emergent Gestural Scores in a Recurrent Neural Network Model of Vowel Harmony}}. Proceedings of the 2021 meeting of the Society for Computation in Linguistics.
\ii 2021. \href{https://en.x-mol.com/paper/article/1395477939799703552}{\itt{Compositional semantics emerges in neural networks solving math problems}} (Jacob Russin, Roland Fernandez, Hamid Palangi, Eric Rosen, Nebojsa Jojic, Paul Smolensky and Jianfeng Gao) To appear in the proceedings of COGSCI 2021.
\ii 2019. \href{https://journals.linguisticsociety.org/proceedings/index.php/amphonology/article/view/4680}{\itt{Learning a model of gradient French liaison}} (with Paul Smolensky and Matthew Goldrick). Proceedings of the Annual Meeting on Phonology, Stony Brook University, October 2019.
\ii  2019. \href{https://journals.linguisticsociety.org/proceedings/index.php/amphonology/article/view/4683}{\itt{Predicting surface forms in complex inflectional paradigms through phonological constraints}}. Supplemental proceedings of the Annual Meeting on Phonology, Stony Brook University, October 2019.
\ii 2019. \href{https://scholarworks.umass.edu/scil/vol2/iss1/12/}{\itt{Learning complex inflectional paradigms through blended gradient inputs}}. Proceedings of the 2019 meeting of the Society for Computation in Linguistics, January 2019, NYU. 
\ii 2018. \href{https://journals.linguisticsociety.org/proceedings/index.php/amphonology/article/view/4571}{\itt{Weak elements make strong predictions:}} \itt{Evidence for gradient input features from Sino-Japanese compound accent}. Supplemental proceedings of the Annual Meeting on Phonology. October 5-7, 2018. UC San Diego.
\ii 2017. \href{http://roa.rutgers.edu/article/view/1729}{\textit{Predicting semi-regular patterns in morphologically complex words}} (An analysis of the semi-regularity of pitch accent in Japanese compound words) Linguistics Vanguard, 4(1). ROA 1339. http://roa.rutgers.edu/article/view/1729
\ii 2016. \href{http://roa.rutgers.edu/content/article/files/1559_rosen_1.pdf}{\textit{Predicting the unpredictable:}} \textit{capturing the apparent semi-regularity of rendaku voicing in Japanese through harmonic grammar}, In proceedings of the 42nd annual meeting of the Berkeley Linguistic Society, February 5-7, 2016. ROA 1299. http://roa.rutgers.edu/article/view/1559
\end{enumerate}

\noindent \bo{Presentations}
%\noindent \bo{Current research under review}
%\bi


\begin{enumerate}[label={[P\arabic*]}]
\ii 2021. \textit{Human-like morphological prediction through implicative relations}. Oral presentation at  the American International Morphology Meeting, August 26-29, 2021.
\ii 2018. \itt{Efficient representation of inflectional paradigms through harmonic grammar}. Poster presented at the Mid-Atlantic Student Colloquium on Speech, Language and Learning, University of Maryland, May 2018.
\ii 2017. \textit{Predicting semi-regular Japanese accent patterns through gradient strengths of inputs} Presented at the workshop ``Strength in Grammar'', University of Leipzig, November 10-12 2017.
\ii 2016. \textit{Japanese pitch accent as a dynamic computational system: global effects through strictly local interactions} Conference paper presented at the Montreal-Ottawa-Laval-Toronto Phonology Workshop, Carleton University, Ottawa, March 19, 2016.
\end{enumerate}

%\ei

%\noindent \bo{Under review}

%\begin{enumerate}[label={[R\arabic*]}]
%\ii \textit{Learning a model of morphologically complex verbal inflection in Amuzgo through real-valued input forms} Submitted to WCCFL.
%\end{enumerate}


%\ii ] \itt{Machine translation between Turkish, English and Inuktitut using a transformer model with tensor product representations} (Sudha Rao, Paul Smolensky, Paul Soulos, Hamid Palangi, Roland Fernandez, Caitlin Smith, Coleman Hayley, Asli Celikyilmaz, Eric Rosen and Jianfeng Gao)

\noindent\bo{Non-refereed research}

\bi
\ii[] 2017. \textit{Geometrical Morphology} Technical report on morphological paradigms in a new framework based on vectors in high-dimensional space. Work in equal joint collaboration with John Goldsmith, University of Chicago. Available on Arxiv at https://arxiv.org/pdf/1703.04481.pdf 
\ei

\noindent \bo{Research prior to 2016}
\bi
\ii[] \textit{Japanese Loanword Accentuation: Epenthesis and Foot Form Interacting through Edge-Interior Alignment}, Proceedings of the Second International Conference on East Asian Linguistics Vancouver, British Columbia, Canada. Nov. 7-9, 2008.\\ 
https://www.sfu.ca/content/dam/sfu/linguistics/Gradlings/SFUWPL/Rosen\_E.pdf

\ii[] \textit{Deriving the accentual behaviour of Japanese suffixes: a new OT account.} Invited talk given at the Third Joint Meeting of the Kansai and Tokyo Phonology Research Groups (PAIK and TCP), February 21-22, 2008, Atami, Japan. 
\ii[] Review of \textit{Voicing in Japanese}, van de Weijer, Jeroen, Kensuke Nanjo, and Tetsuo Nishihara, eds., Mouton de Gruyter. Berlin. \textit{The Phonetician 97: 144-149. 2008}.\\
 http://www.isphs.org/Phonetician/Phonetician\_97-98.pdf
\ii[] \textit{Systematic irregularity in Japanese rendaku: How the grammar mediates patterned lexical exceptions}
Canadian Journal of Linguistics, vol.\ 48. 2003.
\ei

\noindent \textbf{Personal History}
\begin{itemize}
\ii[] Canadian-born. Canadian citizen. 
\ii[] Worked in the field of education before enrolling as a graduate student in the linguistics programme at the University of BC in 1994. 
\ii[] For family reasons, worked part-time at a non-linguistic job throughout graduate study years and full-time during the last year of completing Ph.D.
\ii[] Also for family reasons, worked full-time at a non-linguistic job from 2001 to 2014 after completing Ph.D.
\ii[] Retired from non-linguistic employment in June 2014 and since then have become active in the field again.
\ii[] Was appointed adjunct professor in the Department of Linguistics at the University of British Columbia in August 2016.
\end{itemize}

\noindent \textbf{Academic Background}

\medskip

\begin{tabular}{lll}


B.G.S.\  & Simon Fraser University  &     1975 \\

M.A.\,  Linguistics  &   University of British Columbia  &   1996 \\

Ph.D.\, Linguistics &  University of British Columbia &    2001\\
\end{tabular}


\medskip


\noindent \textbf{Upgrading and Institutes 2014-2015}

\begin{description}
\ii[University of Washington Comp.\ Ling online courses]
Sept.\ 2014 through March 2015: took 3 online courses in computational linguistics:

\begin{itemize}
\ii[] Ling 473: Introduction to Computational Linguistics
\ii[] Ling 570: Natural Language Processing
\ii[] Ling 575: Introduction to Speech Technology
\end{itemize}

\ii[UBC machine learning, Spring 2015] Directed study in machine learning with Mark Schmidt, Associate Professor, Canada Research Chair in Large-Scale Machine Learning and Alfred P.\ Sloan Research Fellow, Department of Computer Science, UBC

\ii[LSA Summer Institute in Chicago, July 6-31, 2015]  Courses taken:

\begin{itemize}
\ii[] Gradient Symbolic Computation (Paul Smolensky and Matt Goldrick)
 \ii[] Unsupervised Learning of Linguistic Structure (John Goldsmith)
\ii[] Computational Phonology (Jeffrey Heinz and Jason Riggle)
\ii[] Computational Psycholinguistics (Roger Levy and Clinton Bicknell)
\end{itemize}
\end{description}

\noindent \textbf{Research positions 2016-2017}

\begin{description}
\ii[UC Santa Cruz] Visiting researcher, Linguistics Research Center, UC Santa Cruz,
January - March 2016.

Sponsor: Armin Mester, Professor, Department of Linguistics.

\ii[U Chicago] Visiting postdoc\footnote{unofficial: my visit was approved by the department but the university was unable to issue me a J-1 visa because of lack of funding.},
University of Chicago, March to June 2016 and October 2016 through February 2017.

Sponsor: John Goldsmith, Edward Carson Waller Distinguished Service Professor, 
Departments of Linguistics, Computer Science and Physical Science Collegiate Division and Humanities Collegiate Division

\hspace{-1cm} \textbf{Ph.\ D.\ Thesis}

\medskip

\noindent\textit{Phonological processes interacting with the lexicon: variable and non-regular effects in
Japanese phonology.} Ph.\ D.\ Dissertation. University of B.\ C.\ 2001. 

\medskip

\noindent Supervisor: Dr.\ Douglas Pulleyblank. \\
Committee members: Dr.\ H.\ Davis, Dr. B.\ Gick. \\ 
University Examiner: Dr.\ J.\ Stemberger. \\
External Examiner: Dr.\ Armin Mester, University of California at Santa Cruz.

\end{description}

\noindent\textbf{Research Assistantship}

\medskip

Sept.\ 1994 - June 1995   Research Assistant, The Syntax and Phonology of Focus

Project Supervisor: Dr.\ Michael Rochemont, Department of Linguistics, UBC

\bigskip

%\noindent\textbf{Teaching: UBC Lingusitics: phonology (200, 311) and syntax (201, 300, 301) courses}
%
%\begin{tabbing}
%                                2000 xxxxx \= Summer Session Term Ixxxx \=  LING 200 \kill
%                                2000    \>        Summer Session Term I  \> LING 200 \\
%                                2000  \>     Winter Session Term I  \>  LING 311 \\
%                                2001   \>    Summer Session Term I  \>  LING 301 \\
%                                2002    \>   Summer Session Term I  \>  LING 300 \\
%                                2002    \>   Summer Session Term I  \>  LING 301 \\
%                                2003    \>   Winter Session Term II \>  LING 201 \\
%                                2003    \>   Summer Session Term II \>  LING 301 \\
%                                2003    \>   Winter Session Term I  \>  LING 200 \\
%                                2004    \>    Summer Session Term I \>  LING 311 \\ 
%                                2005    \>    Summer Session Term I \>  LING 311 \\
%                                2006    \>    Summer Session Term II \>  LING 311 \\
%                                2007    \>    Summer Session Term I \>  LING 201 \\
%                                2007    \>    Summer Session Term II \>  LING 200 \\
%                                2008    \>    Summer Session Term I \>  LING 311 \\
%                                2010    \>    Summer Session Term II \>  LING 300 \\
%                                2012    \>    Summer Session Term II \>  LING 311 \\
%
%\end{tabbing}
%
%\noindent\textbf{Teaching evaluations for the most recent course taught at UBC for which I have evaluations}
%
%\medskip
%
%The following table shows the median, mode and mean scores for the five ``Arts instructor questions'', where `4' represents `agree' and `5', `strongly agree' for Ling 300 in the summer of 2010.
%
%\bigskip
%
%\begin{tabular}{|l|l|l|l|} \hline
%Course evaluation question & Med. & Mode & Mean \\ \hline \hline
%\scriptsize In classes where the size of the class and content of the course &&& \\
%\scriptsize were appropriate,  student participation in class &&& \\
%\scriptsize was encouraged by the instructor.& 5 & 5  & 4.4 \\ \hline
%\scriptsize High standards of achievement were set.&5&5 & 4.8 \\ \hline
%\scriptsize The instructor was generally well prepared for class.&5&5 & 4.3 \\ \hline
%\scriptsize The instructor was readily available to students outside of class, & &&\\
%\scriptsize (e.g., through email, office hours, or by appointment).&5&5 & 4.5 \\ \hline
%\scriptsize The instructor treated students with respect.&5&5 & 4.6 \\ \hline
%\end{tabular}
%
%\bigskip
%
%The following are student comments  posted under Q8: ``Please comment on any aspects, positive or negative, of your instructor's teaching, attitudes to students, class atmosphere, or any other matters affecting the quality of instruction that you consider worthy of note.''
%
%\begin{itemize}
%\item Very good instructor! Fair to students, great teaching style, clear instructions. Always prepared for lectures, never late, assignments are checked on time and returned back with comments which are really valuable and helpful. For any questions we get clear answers, no beating around the bushes. Clear directions and recommendations.
%
%Overall- great responsibility, which makes everything really different from 201 course taken this year.
%
%And he really does show concern for students learning.
%\item Dr.\ Rosen is a very patient instructor. He thoroughly explained concepts to make sure everyone understood them, which was greatly appreciated. However, while his pace was slow, he covered a lot of material in a very short amount of time. Sometimes there were not enough examples to illustrate concepts, which he did show in office hours sometimes. He was readily available to students through office hours and consistent replies on the discussion board. 
%\item Dr.\ Rosen was very respectful to students and treated them fairly. His knowledge for the subject of linguistics is immeasurable- he is brilliant! And his consideration for students' understanding is evident (and appreciated). However, the challenges in the course (the really tough midterm and assignments) made it less gratifying (as a student who was trying to do well) than anticipated. The demand was too high and if Dr.\ Rosen spaced out the material more, had less assignments (11 in total seems too high, especially the amount of time required to complete them), I think the result would be better. With that said, he is a lovely professor, very knowledgable and his communication both online and through office hours is helpful. His communication on the assignments was often hard to read because of handwriting, but nonetheless, present for the graded assignments. Thank you for a good course.
%\item Always answered questions that students had, never dismissed an opinion. :) I felt comfortable raising my hand and participating.
%\item He gave us a lot of assignments, which was both good and bad at the same time. There was a lot of work, but then again this course requires you to practice a lot. He's a hard marker but reasonable. He's very helpful, I appreciate the extra office hours.
%\item Dr.\ Rosen was very good at making himself accessible for questions, whether after class, during office hours, or on the Vista course. He also wrote useful comments on assignments. I appreciated his concern for making sure everything was very fair. It would have been nice if he had a system for reminding himself of things, as he very often asked the class to remind him to do X or Y either in class or on Vista. Dr.\ Rosen was very respectful of students (though I don't think he needs to avoid using any student names in questions; I don't think people are that bothered by it!). I appreciated how Dr.\ Rosen asked for student input when forming assignments or deciding how to use the final minutes of class time or how to use the time in office hours.
%\item Dr.\ Rosen is a very willing, caring, and fair instructor. He continually went above and beyond student expectations to offer his own time to further aid our comprehension of the material.
%\item Dr Rosen is an excellent teacher. He explains concepts fairly clearly and he is always willing to answer questions. He goes out of his way to make sure that people understand the main concepts of the course.
%\end{itemize}

\noindent\textbf{References}

\medskip

\noindent Dr.\ Paul Smolensky, Krieger-Eisenhower Professor,\\
Department of Cognitive Science,\\
Johns Hopkins University,\\
smolensky@jhu.edu, psmo@microsoft.com\\
http://cogsci.jhu.edu/directory/paul-smolensky/


\bigskip

\noindent Dr.\  Matthew Goldrick, Professor,\\
Department of Linguistics,\\
Northwestern University, \\
Evanston, Illinois.\\
matt-goldrick@northwestern.edu\\
http://faculty.wcas.northwestern.edu/matt-goldrick/

\bigskip

\noindent Dr.\ Eric Bakovi\'c, Professor and Chair,\\
Department of Linguistics,\\
University of California, San Diego,\\
ebakovic@ucsd.edu,\\
https://sites.google.com/ucsd.edu/ebakovic/

\end{document}
